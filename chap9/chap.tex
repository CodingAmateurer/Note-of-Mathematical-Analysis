\chapter{积分的计算和积分的极限}

\begin{example}
	(定积分一例七解) 计算$
		I=\int_0^1{\frac{\ln \left( 1+x \right)}{1+x^2}\text{d}x}
	$.
\end{example}

\begin{proof}
	(方法一) 令$x=tanx$,则
	$$
		\int_0^1{\frac{\ln \left( 1+x \right)}{1+x^2}\text{d}x}=\int_0^{\frac{\pi}{4}}{\ln \left( 1+\tan t \right) \text{d}t}=\int_0^{\frac{\pi}{4}}{\ln \left( \sin t+\cos t \right) \text{d}t}-\int_0^{\frac{\pi}{4}}{\ln\cos t\text{d}t}
	$$
	$$
		=\int_0^{\frac{\pi}{4}}{\ln \left[ \sqrt{2}\cos \left( t-\frac{\pi}{4} \right) \right] \text{d}t-\int_0^{\frac{\pi}{4}}{\ln\cos t\text{d}t}}
	$$
	$$
		=\frac{\pi}{8}\ln 2+\int_{-\frac{\pi}{4}}^0{\ln\cos t\text{d}t}-\int_0^{\frac{\pi}{4}}{\ln\cos t\text{d}t}=\frac{\pi}{8}\ln 2
	$$
	\begin{remark}
		最后一步解法用到了偶函数关于原点对称的闭区间上积分的性质.
	\end{remark}
	(方法二) 考虑含参积分$
		I\left( a \right) =\int_0^1{\frac{\ln \left( 1+ax \right)}{1+x^2}\text{d}x}
	$.\\
	显然$I(0)=0$,$ I(1)=I$,且函数$
		\frac{\ln \left( 1+ax \right)}{1+x^2}
	$及$
		\frac{\partial}{\partial a}\left( \frac{\ln \left( 1+ax \right)}{1+x^2} \right) =\frac{x}{\left( 1+x^2 \right) \left( 1+ax \right)}
	$在$R=[0,1]\times [0,1]$上连续,故
	$$
		I'\left( a \right) =\int_0^1{\frac{\partial}{\partial a}\left( \frac{\ln \left( 1+ax \right)}{1+x^2} \right) dx}=\int_0^1{\frac{x}{\left( 1+x^2 \right) \left( 1+ax \right)}dx}
	$$
	$$
		=\frac{1}{1+a^2}\int_0^1{\left( \frac{a+x}{1+x^2}-\frac{a}{1+ax} \right) dx}=\frac{1}{1+a^2}\left[ \frac{a\pi}{4}+\frac{1}{2}\ln 2-\ln \left( 1+a \right) \right]
	$$
	因此
	$$
		I=\int_0^1{I'\left( a \right) \text{d}a=}\int_0^1{\frac{1}{1+a^2}\left[ \frac{a\pi}{4}+\frac{1}{2}\ln 2-\ln \left( 1+a \right) \right] da}=\frac{\pi}{4}\ln 2-I
	$$
	得
	$$
		I=\frac{\pi}{8}\ln 2
	$$

	(方法三) 令$
		x=\frac{1-u}{1+u}
	$,则$
		dx=\frac{-2}{\left( 1+u \right) ^2}du
	$,
	$$
		I=2\int_0^1{\frac{\ln \left( 1+\frac{1-u}{1+u} \right)}{1+\left( \frac{1-u}{1+u} \right) ^2}\frac{1}{\left( 1+u \right) ^2}du}=\int_0^1{\frac{\ln 2-\ln \left( 1+u \right)}{1+u^2}du}
	$$
	$$
		=\ln 2\int_0^1{\frac{du}{1+u^2}}-\int_0^1{\frac{\ln \left( 1+u \right)}{1+u^2}du}=\frac{\pi}{4}\ln 2-I
	$$
	得
	$$
		I=\frac{\pi}{8}\ln 2
	$$
	\begin{remark}
		选取了最常用的三种解法,其他解法暂时不做要求.
	\end{remark}
\end{proof}
