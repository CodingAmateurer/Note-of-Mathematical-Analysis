\chapter{连续}

\section{连续性的证明}

要证明一个函数$f$在某区间$I$上连续,只需要在区间里任意取定一点$x_0\in I$,证明$\lim\limits_{x\rightarrow x_0}f\left( x \right) =f\left( x_0 \right) $.为此,我们可以:

\begin{enumerate}
	\item 利用\textbf{定义},证明:
	      $$
		      \forall \varepsilon >0,\exists \delta >0,\text{当}\left| x-x_0 \right|<\delta \text{时,有}\left| f\left( x \right) -f\left( x_0 \right) \right|<\varepsilon ;
	      $$
	\item 利用\textbf{左右极限},证明:
	      $$f\left( x_0+0 \right) =f\left( x_0 \right) =f\left( x_0-0 \right) ;$$
	\item 利用\textbf{序列语言},证明:
	      $$\forall \left\{ x_n \right\} \rightarrow x_0,\text{有}f\left( x_n \right) \rightarrow f\left( x_0 \right) ;$$
	\item 利用\textbf{邻域语言},证明:
	      $$
		      f\left( \left( x_0-\delta ,x_0+\delta \right) \right) \subset \left( f\left( x_0 \right) -\varepsilon ,f\left( x_0 \right) +\varepsilon \right) ;
	      $$
	\item 利用\textbf{连续函数的运算性质}:连续函数经过有限次$+,-,\times ,\div $( 除法要求除数不为0) ,复合(内层函数的值域在外层函数的定义域内),仍是连续的.
\end{enumerate}

\begin{example}
	证明Riemann函数:
	$$
		R\left( x \right) \left\{ \begin{array}{l}
			\frac{1}{q},\ x=\frac{p}{q}\text{为即约分数,}q>0, \\
			0,\ x\text{为无理数}                             \\
		\end{array} \right.
	$$
	在无理点上连续,在有理点上间断.
\end{example}

\vspace{6cm}

\begin{example}
	设函数$f\left( x \right) $在$\left[ a,b \right] $上单调,且$f\left( \left[ a,b \right] \right) =\left[ f\left( a \right) ,f\left( b \right) \right]$.证明:$f\left( x \right) $在$\left[ a,b \right] $上连续.
\end{example}

\vspace{8cm}

\begin{example}
	设函数$f\left( x \right) $在$\left( a,b \right) $上连续,且$f\left( a+0 \right) =f\left( b-0 \right) =+\infty $.证明:$f\left( x \right) $在$\left( a,b \right) $内能取到最小值.
\end{example}

\vspace{8cm}

\begin{example}
	设$f\left( x \right) $在$\left[ a,b \right] $上连续,证明函数:
	$$
		M\left( x \right) =\underset{a\le t\le x}{sup}f\left( t \right) ,\ m\left( x \right) =\underset{a\le t\le x}{inf}f\left( t \right)
	$$
	在$\left[ a,b \right] $上连续.
\end{example}

\vspace{6cm}

\begin{example}
	讨论函数:
	$$
		f\left( x \right) =\left\{ \begin{array}{l}
			x\left( 1-x \right) ,x\text{为有理数} \\
			x\left( 1+x \right) ,x\text{为无理数} \\
		\end{array} \right.
	$$
	的连续性与可微性.
\end{example}

\vspace{6cm}

\section{闭区间上连续函数的性质}

\begin{theorem}{有界性定理}
	如果函数$f$在闭区间$\left[ a,b \right] $上连续,则$f$在$\left[ a,b \right] $上一定有界.
\end{theorem}

\begin{theorem}{最值定理}
	若函数$f$在闭区间$\left[ a,b \right] $上连续,则$f$在$\left[ a,b \right] $上必有最大值和最小值.
\end{theorem}

\begin{theorem}{介值定理}
	若函数$f$在闭区间$\left[ a,b \right] $上连续,$m$和$M$分别为$f$在闭区间$\left[ a,b \right] $上的最小值和最大值,则对于$m$和$M$之间任意一个实数$c$,至少存在一点$\xi \in \left( a,b \right) $,使得$f\left( \xi \right) =c$.
\end{theorem}

\section{实数的连续性}

\textbf{实数的连续性定理}$($或\textbf{实数的完备性定理}$)$是指以下七个定理的等价:

\begin{theorem}{确界定理}
	任何非空集合$E\subset R$若有下$($上$)$界,则必有下$($上$)$确界$supE(infE)$.
\end{theorem}

\begin{theorem}{单调有界定理}
	单调增加$($减少$)$有下$($上$)$界的数列必收敛,其极限为其下$($上$)$确界.
\end{theorem}

\begin{theorem}{Cauchy准则}
	数列$\left\{ a_n \right\} $收敛的充分必要条件是:对任意的$\varepsilon >0$,存在正整数$N$,当$n,m>N$时,有$\left| a_n-a_m \right|<\varepsilon $.
\end{theorem}

\begin{theorem}{致密性定理}
	有界的数列必有收敛子列.
\end{theorem}

\begin{theorem}{聚点定理}
	有界无穷数集必有聚点.
\end{theorem}

\begin{theorem}{闭区间套定理}
	设$\left[ a_n,b_n \right] $是一串闭区间,且满足下述两个条件:
	$$
		\left( a \right) \left[ a_n,b_n \right] \supset \left[ a_{n+1},b_{n+1} \right] ,n=1,2,\cdots ;\left( b \right) \lim_{n\rightarrow \infty}\left( a_n-b_n \right) =0.
	$$
	则存在唯一的$\xi \in \left[ a,b \right] ,\left( n=1,2,\cdots \right) $且$\lim\limits_{n\rightarrow \infty}a_n=\lim\limits_{n\rightarrow \infty}b_n=\xi $.
\end{theorem}

\begin{note}
	应用闭区间套定理的一般方法是:将题设条件表述成具有性质$P$的闭区间$[a,b]$,并记$[a_1,b_1]=[a,b]$.将$[a_1,b_1]$\textbf{二等分$($有时也可以三等分$)$},等分后得到的两个长度相等的小闭区间应该具有一个性质$P$,取之为$[a_2,b_2]$;将$[a_2,b_2]$二等分,等分后的两个小区间应该具有一个性质$P$,取之为$[a_3,b_3]$;$\cdots \cdots $这样就得到了具有性质$P$的闭区间列${[a_n,b_n]}$.由闭区间定理,必存在唯一的$\xi \in [a_n,b_n]$$($$n=1,2,3,\cdot$$)$,再利用性质$P$推出这个$\xi$就是所要的结论.
\end{note}

\begin{theorem}{有限开覆盖定理}
	设开区间族$\left\{ O_{\lambda}|\lambda \in \varLambda \right\} $满足条件$\underset{\lambda \in \varLambda}{\cup}O_{\lambda}\supset \left[ a,b \right] $,则一定存在该开区间族中的有限个开区间$($不妨设为$)$$O_1,O_2,\cdots ,O_n$,使得$\underset{\lambda \in \varLambda}{\overset{n}{\cup}}O_{\lambda}\supset \left[ a,b \right] $.
\end{theorem}

\begin{note}
	\begin{enumerate}
		\item 有限开覆盖定理有时会被初学者\textbf{错误地理解}为"一个闭区间能够被有限个开区间所覆盖".如果这样理解的话,显然任何一个闭区间只要用一个比它稍大些的开区间就可覆盖了,那么该定理岂不是毫无意义了吗?只要稍加分析,就可以发现这样理解是片面的.因为有限开覆盖定理的本质是"\textbf{覆盖闭区间$[a,b]$的任何一组开区间中,都含有有限个,这有限个开区间也覆盖了$[a,b]$.}"因此,上面说法属于"断章取义",是不全面的.
		\item 有限开覆盖的重要性在于它将无限化为有限,便于从局部性质推出整体性质.
	\end{enumerate}
\end{note}
