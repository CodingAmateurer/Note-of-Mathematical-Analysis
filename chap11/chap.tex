\chapter{积分不等式}

\begin{theorem}{Cauchy-schwarz不等式}
	$$
		\int_0^1{f^2\left( x \right) \text{d}x}\int_0^1{g^2\left( x \right) \text{d}x}\ge \left( \int_0^1{f\left( x \right) g\left( x \right) \text{d}x} \right) ^2
	$$
\end{theorem}

\section{待定系数法证明积分不等式}

\begin{example}
	设$f:[0,1]\rightarrow \mathbb{R}$是Riemann可积函数,满足$
		\int_0^1{xf\left( x \right) \text{d}x}=0
	$.证明:$
		\int_0^1{f^2\left( x \right) \text{d}x}\ge 4\left( \int_0^1{f\left( x \right) \text{d}x} \right) ^2
	$.
\end{example}

\begin{note}
	如果设$p(x)$是多项式,则由Cauchy-schwarz不等式,
	$$
		\int_0^1{f^2\left( x \right) \text{d}x}\int_0^1{p^2\left( x \right) \text{d}x}\ge \left( \int_0^1{f\left( x \right) p\left( x \right) \text{d}x} \right) ^2
	$$,由于$
		\int_0^1{xf\left( x \right) \text{d}x}=0
	$,要保证不等号右边只含$f(x)$,则$p(x)$最多只能是一次多项式,即$deg\ p\left( x \right) =1$,记$
		p\left( x \right) =a_1x+a_0
	$,于是$$
		\int_0^1{p^2\left( x \right) \text{d}x}=\frac{1}{3a_1}\left[ \left( a_1+a_0 \right) ^3-a_{0}^{3} \right]
	$$比较欲证命题的系数可知$$
		\frac{a_{0}^{2}}{4}=\frac{1}{3a_1}\left[ \left( a_1+a_0 \right) ^3-a_{0}^{3} \right]
	$$因此可以让$a_0=-2$,$a_1=3$即可.\\
	同思路题视频传送门:\href{https://www.bilibili.com/video/BV11Q4y1H7QJ?vd_source=0ca98d2b715e9460da2e574e42c1ab8d}{全国大学生数学竞赛非专业组,一类积分不等式难题通解方法}\\
	参考文章传送门:\href{https://zhuanlan.zhihu.com/p/107553208}{数分笔记——待定系数法证积分不等式}
\end{note}

\begin{example}
	设$f \in C^1[0,1]$,满足$f(0)=f(1)=f'(0),f'(1)=1$,则$
		\int_0^1{\left( f''\left( x \right) \right) ^2dx}\ge 4
	$.
\end{example}

\vspace*{7cm}

\begin{example}
	设$f \in C^1[0,1]$,满足$f(0)=f(1)=0$,则$
		\int_0^1{\left( f'\left( x \right) \right) ^2dx}\ge 4\left( \int_0^1{f\left( x \right) \text{d}x} \right) ^2
	$.
\end{example}
